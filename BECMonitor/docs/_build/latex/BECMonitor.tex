% Generated by Sphinx.
\def\sphinxdocclass{report}
\documentclass[letterpaper,10pt,english]{sphinxmanual}
\usepackage[utf8]{inputenc}
\DeclareUnicodeCharacter{00A0}{\nobreakspace}
\usepackage{cmap}
\usepackage[T1]{fontenc}
\usepackage{babel}
\usepackage{times}
\usepackage[Bjarne]{fncychap}
\usepackage{longtable}
\usepackage{sphinx}
\usepackage{multirow}

\addto\captionsenglish{\renewcommand{\figurename}{Fig. }}
\addto\captionsenglish{\renewcommand{\tablename}{Table }}
\floatname{literal-block}{Listing }



\title{BEC Monitor Documentation}
\date{June 04, 2015}
\release{0.0.1}
\author{Zachary Glassman}
\newcommand{\sphinxlogo}{}
\renewcommand{\releasename}{Release}
\makeindex

\makeatletter
\def\PYG@reset{\let\PYG@it=\relax \let\PYG@bf=\relax%
    \let\PYG@ul=\relax \let\PYG@tc=\relax%
    \let\PYG@bc=\relax \let\PYG@ff=\relax}
\def\PYG@tok#1{\csname PYG@tok@#1\endcsname}
\def\PYG@toks#1+{\ifx\relax#1\empty\else%
    \PYG@tok{#1}\expandafter\PYG@toks\fi}
\def\PYG@do#1{\PYG@bc{\PYG@tc{\PYG@ul{%
    \PYG@it{\PYG@bf{\PYG@ff{#1}}}}}}}
\def\PYG#1#2{\PYG@reset\PYG@toks#1+\relax+\PYG@do{#2}}

\expandafter\def\csname PYG@tok@err\endcsname{\def\PYG@bc##1{\setlength{\fboxsep}{0pt}\fcolorbox[rgb]{1.00,0.00,0.00}{1,1,1}{\strut ##1}}}
\expandafter\def\csname PYG@tok@sb\endcsname{\def\PYG@tc##1{\textcolor[rgb]{0.25,0.44,0.63}{##1}}}
\expandafter\def\csname PYG@tok@mf\endcsname{\def\PYG@tc##1{\textcolor[rgb]{0.13,0.50,0.31}{##1}}}
\expandafter\def\csname PYG@tok@na\endcsname{\def\PYG@tc##1{\textcolor[rgb]{0.25,0.44,0.63}{##1}}}
\expandafter\def\csname PYG@tok@s2\endcsname{\def\PYG@tc##1{\textcolor[rgb]{0.25,0.44,0.63}{##1}}}
\expandafter\def\csname PYG@tok@nn\endcsname{\let\PYG@bf=\textbf\def\PYG@tc##1{\textcolor[rgb]{0.05,0.52,0.71}{##1}}}
\expandafter\def\csname PYG@tok@gt\endcsname{\def\PYG@tc##1{\textcolor[rgb]{0.00,0.27,0.87}{##1}}}
\expandafter\def\csname PYG@tok@c1\endcsname{\let\PYG@it=\textit\def\PYG@tc##1{\textcolor[rgb]{0.25,0.50,0.56}{##1}}}
\expandafter\def\csname PYG@tok@go\endcsname{\def\PYG@tc##1{\textcolor[rgb]{0.20,0.20,0.20}{##1}}}
\expandafter\def\csname PYG@tok@cm\endcsname{\let\PYG@it=\textit\def\PYG@tc##1{\textcolor[rgb]{0.25,0.50,0.56}{##1}}}
\expandafter\def\csname PYG@tok@gs\endcsname{\let\PYG@bf=\textbf}
\expandafter\def\csname PYG@tok@ow\endcsname{\let\PYG@bf=\textbf\def\PYG@tc##1{\textcolor[rgb]{0.00,0.44,0.13}{##1}}}
\expandafter\def\csname PYG@tok@gd\endcsname{\def\PYG@tc##1{\textcolor[rgb]{0.63,0.00,0.00}{##1}}}
\expandafter\def\csname PYG@tok@kc\endcsname{\let\PYG@bf=\textbf\def\PYG@tc##1{\textcolor[rgb]{0.00,0.44,0.13}{##1}}}
\expandafter\def\csname PYG@tok@nc\endcsname{\let\PYG@bf=\textbf\def\PYG@tc##1{\textcolor[rgb]{0.05,0.52,0.71}{##1}}}
\expandafter\def\csname PYG@tok@nf\endcsname{\def\PYG@tc##1{\textcolor[rgb]{0.02,0.16,0.49}{##1}}}
\expandafter\def\csname PYG@tok@mb\endcsname{\def\PYG@tc##1{\textcolor[rgb]{0.13,0.50,0.31}{##1}}}
\expandafter\def\csname PYG@tok@vg\endcsname{\def\PYG@tc##1{\textcolor[rgb]{0.73,0.38,0.84}{##1}}}
\expandafter\def\csname PYG@tok@gh\endcsname{\let\PYG@bf=\textbf\def\PYG@tc##1{\textcolor[rgb]{0.00,0.00,0.50}{##1}}}
\expandafter\def\csname PYG@tok@mo\endcsname{\def\PYG@tc##1{\textcolor[rgb]{0.13,0.50,0.31}{##1}}}
\expandafter\def\csname PYG@tok@bp\endcsname{\def\PYG@tc##1{\textcolor[rgb]{0.00,0.44,0.13}{##1}}}
\expandafter\def\csname PYG@tok@vc\endcsname{\def\PYG@tc##1{\textcolor[rgb]{0.73,0.38,0.84}{##1}}}
\expandafter\def\csname PYG@tok@no\endcsname{\def\PYG@tc##1{\textcolor[rgb]{0.38,0.68,0.84}{##1}}}
\expandafter\def\csname PYG@tok@nb\endcsname{\def\PYG@tc##1{\textcolor[rgb]{0.00,0.44,0.13}{##1}}}
\expandafter\def\csname PYG@tok@cp\endcsname{\def\PYG@tc##1{\textcolor[rgb]{0.00,0.44,0.13}{##1}}}
\expandafter\def\csname PYG@tok@gu\endcsname{\let\PYG@bf=\textbf\def\PYG@tc##1{\textcolor[rgb]{0.50,0.00,0.50}{##1}}}
\expandafter\def\csname PYG@tok@gp\endcsname{\let\PYG@bf=\textbf\def\PYG@tc##1{\textcolor[rgb]{0.78,0.36,0.04}{##1}}}
\expandafter\def\csname PYG@tok@cs\endcsname{\def\PYG@tc##1{\textcolor[rgb]{0.25,0.50,0.56}{##1}}\def\PYG@bc##1{\setlength{\fboxsep}{0pt}\colorbox[rgb]{1.00,0.94,0.94}{\strut ##1}}}
\expandafter\def\csname PYG@tok@m\endcsname{\def\PYG@tc##1{\textcolor[rgb]{0.13,0.50,0.31}{##1}}}
\expandafter\def\csname PYG@tok@si\endcsname{\let\PYG@it=\textit\def\PYG@tc##1{\textcolor[rgb]{0.44,0.63,0.82}{##1}}}
\expandafter\def\csname PYG@tok@il\endcsname{\def\PYG@tc##1{\textcolor[rgb]{0.13,0.50,0.31}{##1}}}
\expandafter\def\csname PYG@tok@nd\endcsname{\let\PYG@bf=\textbf\def\PYG@tc##1{\textcolor[rgb]{0.33,0.33,0.33}{##1}}}
\expandafter\def\csname PYG@tok@gr\endcsname{\def\PYG@tc##1{\textcolor[rgb]{1.00,0.00,0.00}{##1}}}
\expandafter\def\csname PYG@tok@ne\endcsname{\def\PYG@tc##1{\textcolor[rgb]{0.00,0.44,0.13}{##1}}}
\expandafter\def\csname PYG@tok@se\endcsname{\let\PYG@bf=\textbf\def\PYG@tc##1{\textcolor[rgb]{0.25,0.44,0.63}{##1}}}
\expandafter\def\csname PYG@tok@nl\endcsname{\let\PYG@bf=\textbf\def\PYG@tc##1{\textcolor[rgb]{0.00,0.13,0.44}{##1}}}
\expandafter\def\csname PYG@tok@w\endcsname{\def\PYG@tc##1{\textcolor[rgb]{0.73,0.73,0.73}{##1}}}
\expandafter\def\csname PYG@tok@kt\endcsname{\def\PYG@tc##1{\textcolor[rgb]{0.56,0.13,0.00}{##1}}}
\expandafter\def\csname PYG@tok@gi\endcsname{\def\PYG@tc##1{\textcolor[rgb]{0.00,0.63,0.00}{##1}}}
\expandafter\def\csname PYG@tok@nv\endcsname{\def\PYG@tc##1{\textcolor[rgb]{0.73,0.38,0.84}{##1}}}
\expandafter\def\csname PYG@tok@k\endcsname{\let\PYG@bf=\textbf\def\PYG@tc##1{\textcolor[rgb]{0.00,0.44,0.13}{##1}}}
\expandafter\def\csname PYG@tok@ss\endcsname{\def\PYG@tc##1{\textcolor[rgb]{0.32,0.47,0.09}{##1}}}
\expandafter\def\csname PYG@tok@ge\endcsname{\let\PYG@it=\textit}
\expandafter\def\csname PYG@tok@ni\endcsname{\let\PYG@bf=\textbf\def\PYG@tc##1{\textcolor[rgb]{0.84,0.33,0.22}{##1}}}
\expandafter\def\csname PYG@tok@sc\endcsname{\def\PYG@tc##1{\textcolor[rgb]{0.25,0.44,0.63}{##1}}}
\expandafter\def\csname PYG@tok@sd\endcsname{\let\PYG@it=\textit\def\PYG@tc##1{\textcolor[rgb]{0.25,0.44,0.63}{##1}}}
\expandafter\def\csname PYG@tok@s1\endcsname{\def\PYG@tc##1{\textcolor[rgb]{0.25,0.44,0.63}{##1}}}
\expandafter\def\csname PYG@tok@mh\endcsname{\def\PYG@tc##1{\textcolor[rgb]{0.13,0.50,0.31}{##1}}}
\expandafter\def\csname PYG@tok@o\endcsname{\def\PYG@tc##1{\textcolor[rgb]{0.40,0.40,0.40}{##1}}}
\expandafter\def\csname PYG@tok@mi\endcsname{\def\PYG@tc##1{\textcolor[rgb]{0.13,0.50,0.31}{##1}}}
\expandafter\def\csname PYG@tok@kd\endcsname{\let\PYG@bf=\textbf\def\PYG@tc##1{\textcolor[rgb]{0.00,0.44,0.13}{##1}}}
\expandafter\def\csname PYG@tok@c\endcsname{\let\PYG@it=\textit\def\PYG@tc##1{\textcolor[rgb]{0.25,0.50,0.56}{##1}}}
\expandafter\def\csname PYG@tok@kr\endcsname{\let\PYG@bf=\textbf\def\PYG@tc##1{\textcolor[rgb]{0.00,0.44,0.13}{##1}}}
\expandafter\def\csname PYG@tok@kn\endcsname{\let\PYG@bf=\textbf\def\PYG@tc##1{\textcolor[rgb]{0.00,0.44,0.13}{##1}}}
\expandafter\def\csname PYG@tok@sx\endcsname{\def\PYG@tc##1{\textcolor[rgb]{0.78,0.36,0.04}{##1}}}
\expandafter\def\csname PYG@tok@sr\endcsname{\def\PYG@tc##1{\textcolor[rgb]{0.14,0.33,0.53}{##1}}}
\expandafter\def\csname PYG@tok@nt\endcsname{\let\PYG@bf=\textbf\def\PYG@tc##1{\textcolor[rgb]{0.02,0.16,0.45}{##1}}}
\expandafter\def\csname PYG@tok@kp\endcsname{\def\PYG@tc##1{\textcolor[rgb]{0.00,0.44,0.13}{##1}}}
\expandafter\def\csname PYG@tok@s\endcsname{\def\PYG@tc##1{\textcolor[rgb]{0.25,0.44,0.63}{##1}}}
\expandafter\def\csname PYG@tok@sh\endcsname{\def\PYG@tc##1{\textcolor[rgb]{0.25,0.44,0.63}{##1}}}
\expandafter\def\csname PYG@tok@vi\endcsname{\def\PYG@tc##1{\textcolor[rgb]{0.73,0.38,0.84}{##1}}}

\def\PYGZbs{\char`\\}
\def\PYGZus{\char`\_}
\def\PYGZob{\char`\{}
\def\PYGZcb{\char`\}}
\def\PYGZca{\char`\^}
\def\PYGZam{\char`\&}
\def\PYGZlt{\char`\<}
\def\PYGZgt{\char`\>}
\def\PYGZsh{\char`\#}
\def\PYGZpc{\char`\%}
\def\PYGZdl{\char`\$}
\def\PYGZhy{\char`\-}
\def\PYGZsq{\char`\'}
\def\PYGZdq{\char`\"}
\def\PYGZti{\char`\~}
% for compatibility with earlier versions
\def\PYGZat{@}
\def\PYGZlb{[}
\def\PYGZrb{]}
\makeatother

\renewcommand\PYGZsq{\textquotesingle}

\begin{document}

\maketitle
\tableofcontents
\phantomsection\label{index::doc}


This software is used in the Lett Lab of the Laser Cooling and
Trapping group at NIST/JQI for experiments on Spinor Na Bose-Einstein
condensates.


\chapter{Usage}
\label{index:bec-monitor}\label{index:usage}
Install the required packages
\begin{itemize}
\item {} 
numpy

\item {} 
scipy

\item {} 
pyqtgraph

\item {} 
pyqt

\item {} 
lmfit

\item {} 
pandas

\end{itemize}

To use just run SpinorMonitor.py

Contents:


\section{Fitobject}
\label{Fitobject::doc}\label{Fitobject:fitobject}
Contents:


\subsection{fit\_object}
\label{fit_object:fit-object}\label{fit_object::doc}\index{fit\_object (class in Fitobject)}

\begin{fulllineitems}
\phantomsection\label{fit_object:Fitobject.fit_object}\pysiglinewithargsret{\strong{class }\code{Fitobject.}\bfcode{fit\_object}}{\emph{index}, \emph{params}, \emph{type\_of\_fit}, \emph{roi}, \emph{data}}{}
fit object holds all the information for a single fit\_sequence
\begin{quote}\begin{description}
\item[{Parameters}] \leavevmode\begin{itemize}
\item {} 
\textbf{\texttt{index}} (\emph{string}) -- Shot number

\item {} 
\textbf{\texttt{params}} (\emph{dictionary}) -- dictionary of Parameters objects containing fit parameters

\item {} 
\textbf{\texttt{type\_of\_fit}} (\emph{string}) -- type of fit to be performed

\item {} 
\textbf{\texttt{roi}} (\emph{list}) -- region of interest to crop data for fit

\item {} 
\textbf{\texttt{data}} (\emph{numpy array}) -- numpy array of image to be analyzed

\end{itemize}

\end{description}\end{quote}
\paragraph{Methods}
\index{BEC\_num() (Fitobject.fit\_object method)}

\begin{fulllineitems}
\phantomsection\label{fit_object:Fitobject.fit_object.BEC_num}\pysiglinewithargsret{\bfcode{BEC\_num}}{\emph{scalex}, \emph{scaley}}{}
get number of BEC atoms from fit from equation
\begin{gather}
\begin{split}N = \left(\frac{2 \pi}{3\lambda^2}\right)\frac{2\pi A}{5}R_x R_y\end{split}\notag
\end{gather}\begin{quote}\begin{description}
\item[{Parameters}] \leavevmode\begin{itemize}
\item {} 
\textbf{\texttt{scalex}} -- x scale of pixel

\item {} 
\textbf{\texttt{scaley}} -- y scale of pixel

\end{itemize}

\item[{Variables}] \leavevmode\begin{itemize}
\item {} 
\textbf{\texttt{A}} -- fitted Thomas-Fermi amplitude

\item {} 
\textbf{\texttt{Rx}} -- fitted Thomas-Fermi x radius

\item {} 
\textbf{\texttt{Ry}} -- fitted Thomas-Fermi y radius

\item {} 
\textbf{\texttt{sigma}} -- optical density

\end{itemize}

\item[{Returns}] \leavevmode
atom number

\end{description}\end{quote}

\end{fulllineitems}

\index{BEC\_num\_1() (Fitobject.fit\_object method)}

\begin{fulllineitems}
\phantomsection\label{fit_object:Fitobject.fit_object.BEC_num_1}\pysiglinewithargsret{\bfcode{BEC\_num\_1}}{\emph{scalex}, \emph{scaley}, \emph{A}, \emph{dx}, \emph{dy}}{}
helper function for BEC num
get number of BEC atoms from fit from equation
\begin{gather}
\begin{split}N = \left(\frac{2 \pi}{3\lambda^2}\right)\frac{2\pi A}{5}R_x R_y\end{split}\notag
\end{gather}\begin{quote}\begin{description}
\item[{Parameters}] \leavevmode\begin{itemize}
\item {} 
\textbf{\texttt{scalex}} -- x scale of pixel

\item {} 
\textbf{\texttt{scaley}} -- y scale of pixel

\item {} 
\textbf{\texttt{A}} -- fitted Thomas-Fermi amplitude

\item {} 
\textbf{\texttt{dx}} -- fitted Thomas-Fermi x radius

\item {} 
\textbf{\texttt{dy}} -- fitted Thomas-Fermi y radius

\end{itemize}

\item[{Variables}] \leavevmode
\textbf{\texttt{sigma}} -- optical density

\item[{Returns}] \leavevmode
atom number

\end{description}\end{quote}

\end{fulllineitems}

\index{TF\_2D() (Fitobject.fit\_object method)}

\begin{fulllineitems}
\phantomsection\label{fit_object:Fitobject.fit_object.TF_2D}\pysiglinewithargsret{\bfcode{TF\_2D}}{}{}
two dimensional Thomas Fermi which is not normalized of the form:
\begin{gather}
\begin{split}TF = A \max\left\{\left[1-\left(\frac{x_c}{dx}\right)^2-\left(\frac{y_c}{dy}\right)^2\right],0\right\}^{3/2}\end{split}\notag
\end{gather}\begin{quote}\begin{description}
\item[{Variables}] \leavevmode\begin{itemize}
\item {} 
\textbf{\texttt{x0}} -- absolute x center

\item {} 
\textbf{\texttt{y0}} -- absolute y center

\item {} 
\textbf{\texttt{xc}} -- rotated x center

\item {} 
\textbf{\texttt{yc}} -- rotated y center

\item {} 
\textbf{\texttt{theta}} -- angle relative to x axis

\item {} 
\textbf{\texttt{A}} -- amplitude

\item {} 
\textbf{\texttt{dx}} -- Thomas-Fermi radius on rotated x axis

\item {} 
\textbf{\texttt{dy}} -- Thomas-Fermi radius on rotated y axis

\item {} 
\textbf{\texttt{off}} -- offsett

\end{itemize}

\end{description}\end{quote}

\end{fulllineitems}

\index{Therm\_num() (Fitobject.fit\_object method)}

\begin{fulllineitems}
\phantomsection\label{fit_object:Fitobject.fit_object.Therm_num}\pysiglinewithargsret{\bfcode{Therm\_num}}{\emph{scalex}, \emph{scaley}}{}
get number of BEC atoms from fit from equation
\begin{gather}
\begin{split}N = \left(\frac{2 \pi}{3\lambda^2}\right)\frac{2\pi A}{5}R_x R_y\end{split}\notag
\end{gather}\begin{quote}\begin{description}
\item[{Parameters}] \leavevmode\begin{itemize}
\item {} 
\textbf{\texttt{scalex}} -- x scale of pixel

\item {} 
\textbf{\texttt{scaley}} -- y scale of pixel

\end{itemize}

\item[{Variables}] \leavevmode\begin{itemize}
\item {} 
\textbf{\texttt{A}} -- fitted Gaussian amplitude

\item {} 
\textbf{\texttt{Rx}} -- fitted Gaussian x standard deviation

\item {} 
\textbf{\texttt{Ry}} -- fitted Gaussian y standard deviation

\item {} 
\textbf{\texttt{sigma}} -- optical density

\end{itemize}

\item[{Returns}] \leavevmode
atom number

\end{description}\end{quote}

\end{fulllineitems}

\index{bimod2min() (Fitobject.fit\_object method)}

\begin{fulllineitems}
\phantomsection\label{fit_object:Fitobject.fit_object.bimod2min}\pysiglinewithargsret{\bfcode{bimod2min}}{\emph{params}}{}
function to minimize, need to subtract offset since included
in both terms

\end{fulllineitems}

\index{create\_vecs() (Fitobject.fit\_object method)}

\begin{fulllineitems}
\phantomsection\label{fit_object:Fitobject.fit_object.create_vecs}\pysiglinewithargsret{\bfcode{create\_vecs}}{\emph{roi}}{}
create vectors scaled by pixel size
\begin{quote}\begin{description}
\item[{Parameters}] \leavevmode
\textbf{\texttt{roi}} (\emph{list}) -- region of interest list

\item[{Return X}] \leavevmode
x vector from meshgrid

\item[{Return Y}] \leavevmode
y vector from meshgrid

\end{description}\end{quote}

\end{fulllineitems}

\index{fit\_image() (Fitobject.fit\_object method)}

\begin{fulllineitems}
\phantomsection\label{fit_object:Fitobject.fit_object.fit_image}\pysiglinewithargsret{\bfcode{fit\_image}}{}{}
fit corrected image with parameters from params

\end{fulllineitems}

\index{gauss\_2D() (Fitobject.fit\_object method)}

\begin{fulllineitems}
\phantomsection\label{fit_object:Fitobject.fit_object.gauss_2D}\pysiglinewithargsret{\bfcode{gauss\_2D}}{}{}
two dimensional Gaussian which is not normalized of the form:
\begin{gather}
\begin{split}G = A \exp\left(-\frac{(x-x_c)^2}{2dx^2}- \frac{(y-y_c)^2}{2dy^2}\right)+ Off\end{split}\notag
\end{gather}\begin{quote}\begin{description}
\item[{Variables}] \leavevmode\begin{itemize}
\item {} 
\textbf{\texttt{x0}} -- absolute x center

\item {} 
\textbf{\texttt{y0}} -- absolute y center

\item {} 
\textbf{\texttt{xc}} -- rotated x center

\item {} 
\textbf{\texttt{yc}} -- rotated y center

\item {} 
\textbf{\texttt{theta}} -- angle relative to x axis

\item {} 
\textbf{\texttt{A}} -- amplitude

\item {} 
\textbf{\texttt{dx}} -- standard deviation on rotated x axis

\item {} 
\textbf{\texttt{dy}} -- standard deviation on rotated y axis

\item {} 
\textbf{\texttt{off}} -- offsett

\end{itemize}

\end{description}\end{quote}

\end{fulllineitems}

\index{get\_angled\_line() (Fitobject.fit\_object method)}

\begin{fulllineitems}
\phantomsection\label{fit_object:Fitobject.fit_object.get_angled_line}\pysiglinewithargsret{\bfcode{get\_angled\_line}}{\emph{x0}, \emph{y0}, \emph{theta}}{}
get angled line for angle theta with formulas
\begin{gather}
\begin{split}x_c = (x-x_0) \cos(\theta) - (y-y_0) \sin(\theta)\end{split}\notag\\\begin{split}y_c = (x-x_0) \sin(\theta) - (y-y_0) \cos(\theta)\end{split}\notag
\end{gather}\begin{quote}\begin{description}
\item[{Parameters}] \leavevmode\begin{itemize}
\item {} 
\textbf{\texttt{x0}} -- absolute x center

\item {} 
\textbf{\texttt{y0}} -- absolute y center

\item {} 
\textbf{\texttt{xc}} -- rotated x center

\item {} 
\textbf{\texttt{yc}} -- rotated y center

\item {} 
\textbf{\texttt{theta}} -- angle relative to x axis

\end{itemize}

\end{description}\end{quote}

\end{fulllineitems}

\index{line\_profile() (Fitobject.fit\_object method)}

\begin{fulllineitems}
\phantomsection\label{fit_object:Fitobject.fit_object.line_profile}\pysiglinewithargsret{\bfcode{line\_profile}}{}{}
calculate line profile, with zeroes to make full image
\begin{quote}\begin{description}
\item[{Returns}] \leavevmode
two-dimensional array which has padding outside of the region of

\end{description}\end{quote}

interest and can be summed for profiles.

\end{fulllineitems}

\index{multiple\_fits() (Fitobject.fit\_object method)}

\begin{fulllineitems}
\phantomsection\label{fit_object:Fitobject.fit_object.multiple_fits}\pysiglinewithargsret{\bfcode{multiple\_fits}}{}{}
function to fit sequentially with input defined from SpinorMonitor
we may need to take parameters of previous fit!!
do fit, update values, do next fit

\end{fulllineitems}

\index{partial\_TF\_2D() (Fitobject.fit\_object method)}

\begin{fulllineitems}
\phantomsection\label{fit_object:Fitobject.fit_object.partial_TF_2D}\pysiglinewithargsret{\bfcode{partial\_TF\_2D}}{\emph{xc}, \emph{yc}, \emph{A}, \emph{dx}, \emph{dy}}{}
two dimensional non-rotated Thomas Fermi which is not normalized of the form:
\begin{gather}
\begin{split}TF = A \max\left\{\left[1-\left(\frac{x_c}{dx}\right)^2-\left(\frac{y_c}{dy}\right)^2\right],0\right\}^{3/2}\end{split}\notag
\end{gather}\begin{quote}\begin{description}
\item[{Parameters}] \leavevmode\begin{itemize}
\item {} 
\textbf{\texttt{xc}} -- absolute x center

\item {} 
\textbf{\texttt{yc}} -- absolute y center

\item {} 
\textbf{\texttt{A}} -- amplitude

\item {} 
\textbf{\texttt{dx}} -- Thomas-Fermi radius on rotated x axis

\item {} 
\textbf{\texttt{dy}} -- Thomas-Fermi radius on rotated y axis

\item {} 
\textbf{\texttt{off}} -- offsett

\end{itemize}

\end{description}\end{quote}

\end{fulllineitems}

\index{process\_results() (Fitobject.fit\_object method)}

\begin{fulllineitems}
\phantomsection\label{fit_object:Fitobject.fit_object.process_results}\pysiglinewithargsret{\bfcode{process\_results}}{\emph{scalex}, \emph{scaley}}{}
process results of fit and allow output return dictonary
scale with the appropriate pixel values after fit

\end{fulllineitems}

\index{sg2min() (Fitobject.fit\_object method)}

\begin{fulllineitems}
\phantomsection\label{fit_object:Fitobject.fit_object.sg2min}\pysiglinewithargsret{\bfcode{sg2min}}{\emph{params}}{}
stern gerlach function to minimize

\end{fulllineitems}

\index{stern\_gerlach\_2D() (Fitobject.fit\_object method)}

\begin{fulllineitems}
\phantomsection\label{fit_object:Fitobject.fit_object.stern_gerlach_2D}\pysiglinewithargsret{\bfcode{stern\_gerlach\_2D}}{}{}
2 dimensional three thomas fermi distributions

\end{fulllineitems}

\index{subtract\_background() (Fitobject.fit\_object method)}

\begin{fulllineitems}
\phantomsection\label{fit_object:Fitobject.fit_object.subtract_background}\pysiglinewithargsret{\bfcode{subtract\_background}}{}{}
Subtract background from image looking at first and last 20
rows of the inital image far away from experiment

\end{fulllineitems}


\end{fulllineitems}



\section{Datatablewidget}
\label{Datatablewidget::doc}\label{Datatablewidget:datatablewidget}
Contents:


\subsection{DataTable}
\label{DataTable::doc}\label{DataTable:datatable}\index{DataTable (class in Datatablewidget)}

\begin{fulllineitems}
\phantomsection\label{DataTable:Datatablewidget.DataTable}\pysiglinewithargsret{\strong{class }\code{Datatablewidget.}\bfcode{DataTable}}{\emph{parent=None}}{}
tabbed tables to show system parameters and fitted parameters
\paragraph{Methods}
\index{update\_pandas\_table() (Datatablewidget.DataTable method)}

\begin{fulllineitems}
\phantomsection\label{DataTable:Datatablewidget.DataTable.update_pandas_table}\pysiglinewithargsret{\bfcode{update\_pandas\_table}}{\emph{df}}{}
update tables, check if cols are different

\end{fulllineitems}


\end{fulllineitems}



\section{Auxfuncwidget}
\label{Auxfuncwidget::doc}\label{Auxfuncwidget:auxfuncwidget}
Contents:


\subsection{AuxillaryFunctionContainerWidget}
\label{AuxillaryFunctionContainerWidget::doc}\label{AuxillaryFunctionContainerWidget:auxillaryfunctioncontainerwidget}\index{AuxillaryFunctionContainerWidget (class in Auxfuncwidget)}

\begin{fulllineitems}
\phantomsection\label{AuxillaryFunctionContainerWidget:Auxfuncwidget.AuxillaryFunctionContainerWidget}\pysiglinewithargsret{\strong{class }\code{Auxfuncwidget.}\bfcode{AuxillaryFunctionContainerWidget}}{\emph{parent=None}}{}
class for displaying container of auxillary function widgets
will hold a stacked layout of all auxillary functions
\paragraph{Methods}
\index{add\_element() (Auxfuncwidget.AuxillaryFunctionContainerWidget method)}

\begin{fulllineitems}
\phantomsection\label{AuxillaryFunctionContainerWidget:Auxfuncwidget.AuxillaryFunctionContainerWidget.add_element}\pysiglinewithargsret{\bfcode{add\_element}}{\emph{name}}{}
convenicne function to create function widget and add to proper
dictionaries

\end{fulllineitems}

\index{re\_import() (Auxfuncwidget.AuxillaryFunctionContainerWidget method)}

\begin{fulllineitems}
\phantomsection\label{AuxillaryFunctionContainerWidget:Auxfuncwidget.AuxillaryFunctionContainerWidget.re_import}\pysiglinewithargsret{\bfcode{re\_import}}{}{}
\end{fulllineitems}


\end{fulllineitems}



\subsection{AuxillaryFunctionWidget}
\label{AuxillaryFunctionWidget::doc}\label{AuxillaryFunctionWidget:auxillaryfunctionwidget}\index{AuxillaryFunctionWidget (class in Auxfuncwidget)}

\begin{fulllineitems}
\phantomsection\label{AuxillaryFunctionWidget:Auxfuncwidget.AuxillaryFunctionWidget}\pysiglinewithargsret{\strong{class }\code{Auxfuncwidget.}\bfcode{AuxillaryFunctionWidget}}{\emph{func}, \emph{parent=None}}{}
class holding function and entry information
\paragraph{Methods}
\index{calculate() (Auxfuncwidget.AuxillaryFunctionWidget method)}

\begin{fulllineitems}
\phantomsection\label{AuxillaryFunctionWidget:Auxfuncwidget.AuxillaryFunctionWidget.calculate}\pysiglinewithargsret{\bfcode{calculate}}{}{}
\end{fulllineitems}

\index{generate\_info\_widgets() (Auxfuncwidget.AuxillaryFunctionWidget method)}

\begin{fulllineitems}
\phantomsection\label{AuxillaryFunctionWidget:Auxfuncwidget.AuxillaryFunctionWidget.generate_info_widgets}\pysiglinewithargsret{\bfcode{generate\_info\_widgets}}{}{}
generate info sublayouts

\end{fulllineitems}

\index{generate\_params\_widgets() (Auxfuncwidget.AuxillaryFunctionWidget method)}

\begin{fulllineitems}
\phantomsection\label{AuxillaryFunctionWidget:Auxfuncwidget.AuxillaryFunctionWidget.generate_params_widgets}\pysiglinewithargsret{\bfcode{generate\_params\_widgets}}{}{}
generate parameter sublayout and return layout

\end{fulllineitems}

\index{get\_params() (Auxfuncwidget.AuxillaryFunctionWidget method)}

\begin{fulllineitems}
\phantomsection\label{AuxillaryFunctionWidget:Auxfuncwidget.AuxillaryFunctionWidget.get_params}\pysiglinewithargsret{\bfcode{get\_params}}{}{}
\end{fulllineitems}


\end{fulllineitems}



\section{Ipython}
\label{Ipython::doc}\label{Ipython:ipython}
Contents:


\subsection{PlotObj}
\label{PlotObj:plotobj}\label{PlotObj::doc}\index{PlotObj (class in Ipython)}

\begin{fulllineitems}
\phantomsection\label{PlotObj:Ipython.PlotObj}\pysigline{\strong{class }\code{Ipython.}\bfcode{PlotObj}}
class to hold  SpinorPlot objects
\paragraph{Methods}
\index{add\_plot() (Ipython.PlotObj method)}

\begin{fulllineitems}
\phantomsection\label{PlotObj:Ipython.PlotObj.add_plot}\pysiglinewithargsret{\bfcode{add\_plot}}{\emph{plot}, \emph{name}}{}
Add plot to dictionary of plots to update

\end{fulllineitems}

\index{update() (Ipython.PlotObj method)}

\begin{fulllineitems}
\phantomsection\label{PlotObj:Ipython.PlotObj.update}\pysiglinewithargsret{\bfcode{update}}{\emph{var\_dict}}{}
update all plots in dictionary

\end{fulllineitems}


\end{fulllineitems}



\subsection{QIPythonWidget}
\label{QIPythonWidget::doc}\label{QIPythonWidget:qipythonwidget}\index{QIPythonWidget (class in Ipython)}

\begin{fulllineitems}
\phantomsection\label{QIPythonWidget:Ipython.QIPythonWidget}\pysiglinewithargsret{\strong{class }\code{Ipython.}\bfcode{QIPythonWidget}}{\emph{customBanner=None}, \emph{*args}, \emph{**kwargs}}{}
Convenience class for a live IPython console widget.
This widget lives within the main GUI
\paragraph{Attributes}

\begin{tabulary}{\linewidth}{|L|L|}
\hline

custom\_control
 & \\
\hline
custom\_page\_control
 & \\
\hline\end{tabulary}

\paragraph{Methods}
\index{clearTerminal() (Ipython.QIPythonWidget method)}

\begin{fulllineitems}
\phantomsection\label{QIPythonWidget:Ipython.QIPythonWidget.clearTerminal}\pysiglinewithargsret{\bfcode{clearTerminal}}{}{}
Clears the terminal

\end{fulllineitems}

\index{executeCommand() (Ipython.QIPythonWidget method)}

\begin{fulllineitems}
\phantomsection\label{QIPythonWidget:Ipython.QIPythonWidget.executeCommand}\pysiglinewithargsret{\bfcode{executeCommand}}{\emph{command}}{}
Execute a command in the frame of the console widget

\end{fulllineitems}

\index{printText() (Ipython.QIPythonWidget method)}

\begin{fulllineitems}
\phantomsection\label{QIPythonWidget:Ipython.QIPythonWidget.printText}\pysiglinewithargsret{\bfcode{printText}}{\emph{text}}{}
Prints some plain text to the console

\end{fulllineitems}

\index{pushVariables() (Ipython.QIPythonWidget method)}

\begin{fulllineitems}
\phantomsection\label{QIPythonWidget:Ipython.QIPythonWidget.pushVariables}\pysiglinewithargsret{\bfcode{pushVariables}}{\emph{variableDict}}{}
Given a dictionary containing name / value pairs, 
push those variables to the IPython console widget

\end{fulllineitems}


\end{fulllineitems}



\subsection{QIPythonWidgetContainer}
\label{QIPythonWidgetContainer::doc}\label{QIPythonWidgetContainer:qipythonwidgetcontainer}\index{QIPythonWidgetContainer (class in Ipython)}

\begin{fulllineitems}
\phantomsection\label{QIPythonWidgetContainer:Ipython.QIPythonWidgetContainer}\pysiglinewithargsret{\strong{class }\code{Ipython.}\bfcode{QIPythonWidgetContainer}}{\emph{parent=None}}{}
Ipython container class for multi-threading
\paragraph{Methods}

\end{fulllineitems}



\subsection{SpinorPlot}
\label{SpinorPlot::doc}\label{SpinorPlot:spinorplot}\index{SpinorPlot (class in Ipython)}

\begin{fulllineitems}
\phantomsection\label{SpinorPlot:Ipython.SpinorPlot}\pysiglinewithargsret{\strong{class }\code{Ipython.}\bfcode{SpinorPlot}}{\emph{func}, \emph{name=None}, \emph{xaxis=None}, \emph{yaxis=None}}{}
class to plot with updating stuff
\paragraph{Methods}
\index{get\_vars() (Ipython.SpinorPlot method)}

\begin{fulllineitems}
\phantomsection\label{SpinorPlot:Ipython.SpinorPlot.get_vars}\pysiglinewithargsret{\bfcode{get\_vars}}{\emph{var\_dict}}{}
\end{fulllineitems}

\index{set\_axis() (Ipython.SpinorPlot method)}

\begin{fulllineitems}
\phantomsection\label{SpinorPlot:Ipython.SpinorPlot.set_axis}\pysiglinewithargsret{\bfcode{set\_axis}}{}{}
\end{fulllineitems}

\index{update\_plot() (Ipython.SpinorPlot method)}

\begin{fulllineitems}
\phantomsection\label{SpinorPlot:Ipython.SpinorPlot.update_plot}\pysiglinewithargsret{\bfcode{update\_plot}}{\emph{var\_dict}}{}
\end{fulllineitems}


\end{fulllineitems}



\section{Auxwidgets}
\label{Auxwidgets::doc}\label{Auxwidgets:auxwidgets}
Contents:


\subsection{FingerTabBarWidget}
\label{FingerTabBarWidget::doc}\label{FingerTabBarWidget:fingertabbarwidget}\index{FingerTabBarWidget (class in Auxwidgets)}

\begin{fulllineitems}
\phantomsection\label{FingerTabBarWidget:Auxwidgets.FingerTabBarWidget}\pysiglinewithargsret{\strong{class }\code{Auxwidgets.}\bfcode{FingerTabBarWidget}}{\emph{parent=None}, \emph{*args}, \emph{**kwargs}}{}
Class to implement tabbed browsing for options
\paragraph{Methods}
\index{paintEvent() (Auxwidgets.FingerTabBarWidget method)}

\begin{fulllineitems}
\phantomsection\label{FingerTabBarWidget:Auxwidgets.FingerTabBarWidget.paintEvent}\pysiglinewithargsret{\bfcode{paintEvent}}{\emph{event}}{}
\end{fulllineitems}

\index{tabSizeHint() (Auxwidgets.FingerTabBarWidget method)}

\begin{fulllineitems}
\phantomsection\label{FingerTabBarWidget:Auxwidgets.FingerTabBarWidget.tabSizeHint}\pysiglinewithargsret{\bfcode{tabSizeHint}}{\emph{index}}{}
\end{fulllineitems}


\end{fulllineitems}



\subsection{TextBox}
\label{TextBox::doc}\label{TextBox:textbox}\index{TextBox (class in Auxwidgets)}

\begin{fulllineitems}
\phantomsection\label{TextBox:Auxwidgets.TextBox}\pysigline{\strong{class }\code{Auxwidgets.}\bfcode{TextBox}}
custom textbox, mostly QTextEdit, with some added functions
\paragraph{Methods}
\index{output() (Auxwidgets.TextBox method)}

\begin{fulllineitems}
\phantomsection\label{TextBox:Auxwidgets.TextBox.output}\pysiglinewithargsret{\bfcode{output}}{\emph{x}}{}
\end{fulllineitems}


\end{fulllineitems}



\section{Subroutines}
\label{Subroutines::doc}\label{Subroutines:subroutines}
Contents:


\section{Dataplots}
\label{Dataplots::doc}\label{Dataplots:dataplots}
Contents:


\subsection{ImageWindow}
\label{ImageWindow::doc}\label{ImageWindow:imagewindow}\index{ImageWindow (class in Dataplots)}

\begin{fulllineitems}
\phantomsection\label{ImageWindow:Dataplots.ImageWindow}\pysiglinewithargsret{\strong{class }\code{Dataplots.}\bfcode{ImageWindow}}{\emph{parent=None}}{}
Image View with custom ROI
\paragraph{Attributes}

\begin{tabulary}{\linewidth}{|L|L|}
\hline

lastFileDir
 & \\
\hline\end{tabulary}

\paragraph{Methods}
\index{add\_lines() (Dataplots.ImageWindow method)}

\begin{fulllineitems}
\phantomsection\label{ImageWindow:Dataplots.ImageWindow.add_lines}\pysiglinewithargsret{\bfcode{add\_lines}}{\emph{results}}{}
add lines to plot, input it numpy array which is then summed

\end{fulllineitems}

\index{popup() (Dataplots.ImageWindow method)}

\begin{fulllineitems}
\phantomsection\label{ImageWindow:Dataplots.ImageWindow.popup}\pysiglinewithargsret{\bfcode{popup}}{}{}
function to start popup window object

\end{fulllineitems}

\index{setImage() (Dataplots.ImageWindow method)}

\begin{fulllineitems}
\phantomsection\label{ImageWindow:Dataplots.ImageWindow.setImage}\pysiglinewithargsret{\bfcode{setImage}}{\emph{im}}{}
set image

\end{fulllineitems}

\index{updatePlot() (Dataplots.ImageWindow method)}

\begin{fulllineitems}
\phantomsection\label{ImageWindow:Dataplots.ImageWindow.updatePlot}\pysiglinewithargsret{\bfcode{updatePlot}}{}{}
updates plot, can only be called once plot is initalized with image

\end{fulllineitems}


\end{fulllineitems}



\subsection{PlotGrid}
\label{PlotGrid::doc}\label{PlotGrid:plotgrid}\index{PlotGrid (class in Dataplots)}

\begin{fulllineitems}
\phantomsection\label{PlotGrid:Dataplots.PlotGrid}\pysiglinewithargsret{\strong{class }\code{Dataplots.}\bfcode{PlotGrid}}{\emph{parent=None}}{}~\paragraph{Methods}

\end{fulllineitems}



\section{Fitmodels}
\label{Fitmodels::doc}\label{Fitmodels:fitmodels}
Contents:


\section{Image}
\label{Image::doc}\label{Image:image}
Contents:


\subsection{IncomingImage}
\label{IncomingImage::doc}\label{IncomingImage:incomingimage}\index{IncomingImage (class in Image)}

\begin{fulllineitems}
\phantomsection\label{IncomingImage:Image.IncomingImage}\pysigline{\strong{class }\code{Image.}\bfcode{IncomingImage}}
check for images, if found obtain image and send back to main GUI
\paragraph{Methods}
\index{newImage() (Image.IncomingImage method)}

\begin{fulllineitems}
\phantomsection\label{IncomingImage:Image.IncomingImage.newImage}\pysiglinewithargsret{\bfcode{newImage}}{}{}
This function checks in directory for new image with proper name
if found, it reads it in and then deletes it

\end{fulllineitems}

\index{run() (Image.IncomingImage method)}

\begin{fulllineitems}
\phantomsection\label{IncomingImage:Image.IncomingImage.run}\pysiglinewithargsret{\bfcode{run}}{}{}
every second search folder for new images, if found
get image and emit back to main gui for processing

\end{fulllineitems}


\end{fulllineitems}



\subsection{ProcessImage}
\label{ProcessImage::doc}\label{ProcessImage:processimage}\index{ProcessImage (class in Image)}

\begin{fulllineitems}
\phantomsection\label{ProcessImage:Image.ProcessImage}\pysiglinewithargsret{\strong{class }\code{Image.}\bfcode{ProcessImage}}{\emph{data}, \emph{exp\_data}, \emph{options}, \emph{path}, \emph{run}}{}
Processing object for threading purposes
@parameters
\begin{quote}

data: numpy array
options: list of options for fit parameters
\begin{quote}
\begin{quote}

{[}params,
\end{quote}

type\_of\_fit,
ROI,
index
\end{quote}
\end{quote}
\paragraph{Methods}
\index{run() (Image.ProcessImage method)}

\begin{fulllineitems}
\phantomsection\label{ProcessImage:Image.ProcessImage.run}\pysiglinewithargsret{\bfcode{run}}{}{}
process results using methods from fit process and emit

\end{fulllineitems}


\end{fulllineitems}



\section{SpinorMonitor}
\label{SpinorMonitor::doc}\label{SpinorMonitor:spinormonitor}
Contents:


\subsection{MainWindow}
\label{MainWindow::doc}\label{MainWindow:mainwindow}\index{MainWindow (class in SpinorMonitor)}

\begin{fulllineitems}
\phantomsection\label{MainWindow:SpinorMonitor.MainWindow}\pysigline{\strong{class }\code{SpinorMonitor.}\bfcode{MainWindow}}
Main Window for the app, contains the graphs panel and the options
panel.  Executes main control of all other panels.
\begin{quote}\begin{description}
\item[{Variables}] \leavevmode\begin{itemize}
\item {} 
\textbf{\texttt{expData}} -- Pandas dataframe where all experiment information is kept

\item {} 
{\hyperref[ProcessImage:Image.ProcessImage.run]{\emph{\textbf{\texttt{run}}}}} -- Run number for the day

\item {} 
\textbf{\texttt{path}} -- Path to data storage folder

\item {} 
\textbf{\texttt{processThreadPool}} -- Dictionary of running threads

\item {} 
\textbf{\texttt{process}} -- Convenience dictionary to initialize objects

\item {} 
\textbf{\texttt{ROI}} -- region of interest

\item {} 
\textbf{\texttt{running}} -- Boolean if data collection thread is active

\item {} 
\textbf{\texttt{index}} -- keeps track of shot internally

\item {} 
\textbf{\texttt{image}} -- ImageWindow widget

\item {} 
\textbf{\texttt{plots}} -- DataPlots widget

\item {} 
\textbf{\texttt{options}} -- Options widget

\item {} 
\textbf{\texttt{plot\_options}} -- PlotOptions widget

\item {} 
\textbf{\texttt{vis\_plots}} -- VisualPlotter widget

\item {} 
\textbf{\texttt{data\_tables}} -- DataTable widget

\item {} 
\textbf{\texttt{aux\_funcs}} -- AuxillaryFunctionContainerWidget widget

\item {} 
\textbf{\texttt{tabs}} -- QTabWidget, contains other widgets

\item {} 
\textbf{\texttt{ipy}} -- QIPythonWidget

\end{itemize}

\end{description}\end{quote}
\paragraph{Methods}
\index{center() (SpinorMonitor.MainWindow method)}

\begin{fulllineitems}
\phantomsection\label{MainWindow:SpinorMonitor.MainWindow.center}\pysiglinewithargsret{\bfcode{center}}{}{}
Centers Window

\end{fulllineitems}

\index{change\_state() (SpinorMonitor.MainWindow method)}

\begin{fulllineitems}
\phantomsection\label{MainWindow:SpinorMonitor.MainWindow.change_state}\pysiglinewithargsret{\bfcode{change\_state}}{}{}
start and stop data collection thread

\end{fulllineitems}

\index{data\_process() (SpinorMonitor.MainWindow method)}

\begin{fulllineitems}
\phantomsection\label{MainWindow:SpinorMonitor.MainWindow.data_process}\pysiglinewithargsret{\bfcode{data\_process}}{\emph{results\_dict}}{}
process the data, including spawn a thread and increment index

\end{fulllineitems}

\index{data\_recieved() (SpinorMonitor.MainWindow method)}

\begin{fulllineitems}
\phantomsection\label{MainWindow:SpinorMonitor.MainWindow.data_recieved}\pysiglinewithargsret{\bfcode{data\_recieved}}{}{}
Send message that data was recieved

\end{fulllineitems}

\index{end() (SpinorMonitor.MainWindow method)}

\begin{fulllineitems}
\phantomsection\label{MainWindow:SpinorMonitor.MainWindow.end}\pysiglinewithargsret{\bfcode{end}}{}{}
function to stop listening Thread, writes out expData to csv
in smae folder as data printing

\end{fulllineitems}

\index{finish\_thread() (SpinorMonitor.MainWindow method)}

\begin{fulllineitems}
\phantomsection\label{MainWindow:SpinorMonitor.MainWindow.finish_thread}\pysiglinewithargsret{\bfcode{finish\_thread}}{\emph{ind}}{}
pop the process should destroy it all I think/

\end{fulllineitems}

\index{get\_options() (SpinorMonitor.MainWindow method)}

\begin{fulllineitems}
\phantomsection\label{MainWindow:SpinorMonitor.MainWindow.get_options}\pysiglinewithargsret{\bfcode{get\_options}}{}{}
convenience function to return list of options
note that function which recieves params must make
deep copy or there will be problems!!

\end{fulllineitems}

\index{get\_roi() (SpinorMonitor.MainWindow method)}

\begin{fulllineitems}
\phantomsection\label{MainWindow:SpinorMonitor.MainWindow.get_roi}\pysiglinewithargsret{\bfcode{get\_roi}}{}{}
returns region of interest in list
:returns: {[}xstart,xend,ystart,yend,angle{]}
:rtype: list

\end{fulllineitems}

\index{initUI() (SpinorMonitor.MainWindow method)}

\begin{fulllineitems}
\phantomsection\label{MainWindow:SpinorMonitor.MainWindow.initUI}\pysiglinewithargsret{\bfcode{initUI}}{}{}
Iniitalize UI and name it.  Creat all children widgets and place 
them in layout

\end{fulllineitems}

\index{on\_fit\_name() (SpinorMonitor.MainWindow method)}

\begin{fulllineitems}
\phantomsection\label{MainWindow:SpinorMonitor.MainWindow.on_fit_name}\pysiglinewithargsret{\bfcode{on\_fit\_name}}{\emph{data}}{}
Triggers the plots.change\_key functions with argument data.
\begin{quote}\begin{description}
\item[{Params data}] \leavevmode
name of fit

\end{description}\end{quote}

\end{fulllineitems}

\index{on\_message() (SpinorMonitor.MainWindow method)}

\begin{fulllineitems}
\phantomsection\label{MainWindow:SpinorMonitor.MainWindow.on_message}\pysiglinewithargsret{\bfcode{on\_message}}{\emph{data}}{}
Send message to output windows
\begin{quote}\begin{description}
\item[{Parameters}] \leavevmode
\textbf{\texttt{data}} (\emph{object}) -- message to send

\end{description}\end{quote}

\end{fulllineitems}

\index{set\_up\_ipy() (SpinorMonitor.MainWindow method)}

\begin{fulllineitems}
\phantomsection\label{MainWindow:SpinorMonitor.MainWindow.set_up_ipy}\pysiglinewithargsret{\bfcode{set\_up\_ipy}}{}{}
setup the ipython console for use with useful functions

\end{fulllineitems}

\index{start() (SpinorMonitor.MainWindow method)}

\begin{fulllineitems}
\phantomsection\label{MainWindow:SpinorMonitor.MainWindow.start}\pysiglinewithargsret{\bfcode{start}}{}{}
Function to start listening thread, connect signals and 
:var imageThread: IncomingImage object listening for images

\end{fulllineitems}

\index{to\_ipy() (SpinorMonitor.MainWindow method)}

\begin{fulllineitems}
\phantomsection\label{MainWindow:SpinorMonitor.MainWindow.to_ipy}\pysiglinewithargsret{\bfcode{to\_ipy}}{}{}
push all variables to Ipython notebook

\end{fulllineitems}

\index{update\_data() (SpinorMonitor.MainWindow method)}

\begin{fulllineitems}
\phantomsection\label{MainWindow:SpinorMonitor.MainWindow.update_data}\pysiglinewithargsret{\bfcode{update\_data}}{\emph{results\_passed}}{}
function to update plots and push data to ipython notebook

\end{fulllineitems}


\end{fulllineitems}



\section{Optionswidgets}
\label{Optionswidgets::doc}\label{Optionswidgets:optionswidgets}
Contents:


\subsection{FitInfo}
\label{FitInfo::doc}\label{FitInfo:fitinfo}\index{FitInfo (class in Optionswidgets)}

\begin{fulllineitems}
\phantomsection\label{FitInfo:Optionswidgets.FitInfo}\pysiglinewithargsret{\strong{class }\code{Optionswidgets.}\bfcode{FitInfo}}{\emph{params}, \emph{parent=None}}{}
custom dialog for fit information
\paragraph{Methods}
\index{close() (Optionswidgets.FitInfo method)}

\begin{fulllineitems}
\phantomsection\label{FitInfo:Optionswidgets.FitInfo.close}\pysiglinewithargsret{\bfcode{close}}{}{}
\end{fulllineitems}

\index{parse\_params() (Optionswidgets.FitInfo method)}

\begin{fulllineitems}
\phantomsection\label{FitInfo:Optionswidgets.FitInfo.parse_params}\pysiglinewithargsret{\bfcode{parse\_params}}{\emph{tabs}}{}
populates the tables, row and column determined by run and
parameter, so same for all table

\end{fulllineitems}


\end{fulllineitems}



\subsection{Options}
\label{Options::doc}\label{Options:options}\index{Options (class in Optionswidgets)}

\begin{fulllineitems}
\phantomsection\label{Options:Optionswidgets.Options}\pysiglinewithargsret{\strong{class }\code{Optionswidgets.}\bfcode{Options}}{\emph{parent=None}}{}
Panel which defines options for fitting and analyzing images
\paragraph{Methods}
\index{create\_fit\_panel() (Optionswidgets.Options method)}

\begin{fulllineitems}
\phantomsection\label{Options:Optionswidgets.Options.create_fit_panel}\pysiglinewithargsret{\bfcode{create\_fit\_panel}}{}{}
create a fit panel

\end{fulllineitems}

\index{fit\_name (Optionswidgets.Options attribute)}

\begin{fulllineitems}
\phantomsection\label{Options:Optionswidgets.Options.fit_name}\pysigline{\bfcode{fit\_name}}
\end{fulllineitems}

\index{get\_fit\_info() (Optionswidgets.Options method)}

\begin{fulllineitems}
\phantomsection\label{Options:Optionswidgets.Options.get_fit_info}\pysiglinewithargsret{\bfcode{get\_fit\_info}}{}{}
popup window which has info of all fits

\end{fulllineitems}

\index{make\_key() (Optionswidgets.Options method)}

\begin{fulllineitems}
\phantomsection\label{Options:Optionswidgets.Options.make_key}\pysiglinewithargsret{\bfcode{make\_key}}{\emph{index}}{}
\end{fulllineitems}

\index{message (Optionswidgets.Options attribute)}

\begin{fulllineitems}
\phantomsection\label{Options:Optionswidgets.Options.message}\pysigline{\bfcode{message}}
\end{fulllineitems}

\index{remove\_fit\_panel() (Optionswidgets.Options method)}

\begin{fulllineitems}
\phantomsection\label{Options:Optionswidgets.Options.remove_fit_panel}\pysiglinewithargsret{\bfcode{remove\_fit\_panel}}{}{}
remove fit panel

\end{fulllineitems}

\index{save\_params() (Optionswidgets.Options method)}

\begin{fulllineitems}
\phantomsection\label{Options:Optionswidgets.Options.save_params}\pysiglinewithargsret{\bfcode{save\_params}}{}{}
update params

\end{fulllineitems}

\index{set\_current\_fit() (Optionswidgets.Options method)}

\begin{fulllineitems}
\phantomsection\label{Options:Optionswidgets.Options.set_current_fit}\pysiglinewithargsret{\bfcode{set\_current\_fit}}{\emph{fit\_name}}{}
\end{fulllineitems}


\end{fulllineitems}



\subsection{ParameterEntry}
\label{ParameterEntry::doc}\label{ParameterEntry:parameterentry}\index{ParameterEntry (class in Optionswidgets)}

\begin{fulllineitems}
\phantomsection\label{ParameterEntry:Optionswidgets.ParameterEntry}\pysiglinewithargsret{\strong{class }\code{Optionswidgets.}\bfcode{ParameterEntry}}{\emph{params}, \emph{first}, \emph{parent=None}}{}
popup box to select parameters
\paragraph{Methods}
\index{readout() (Optionswidgets.ParameterEntry method)}

\begin{fulllineitems}
\phantomsection\label{ParameterEntry:Optionswidgets.ParameterEntry.readout}\pysiglinewithargsret{\bfcode{readout}}{}{}
function to return updated Parameters object

\end{fulllineitems}


\end{fulllineitems}



\subsection{PlotOptions}
\label{PlotOptions::doc}\label{PlotOptions:plotoptions}\index{PlotOptions (class in Optionswidgets)}

\begin{fulllineitems}
\phantomsection\label{PlotOptions:Optionswidgets.PlotOptions}\pysiglinewithargsret{\strong{class }\code{Optionswidgets.}\bfcode{PlotOptions}}{\emph{parent=None}}{}
Widget for Region of Interest Information and other plot options
\paragraph{Methods}
\index{set\_roi() (Optionswidgets.PlotOptions method)}

\begin{fulllineitems}
\phantomsection\label{PlotOptions:Optionswidgets.PlotOptions.set_roi}\pysiglinewithargsret{\bfcode{set\_roi}}{\emph{vec}}{}
Generate roi strings and print coords

\end{fulllineitems}


\end{fulllineitems}



\section{Auxfunctions}
\label{Auxfunctions::doc}\label{Auxfunctions:auxfunctions}
Contents:


\section{Visualplotterwidget}
\label{Visualplotterwidget::doc}\label{Visualplotterwidget:visualplotterwidget}
Contents:


\subsection{ParamEntry}
\label{ParamEntry::doc}\label{ParamEntry:paramentry}\index{ParamEntry (class in Visualplotterwidget)}

\begin{fulllineitems}
\phantomsection\label{ParamEntry:Visualplotterwidget.ParamEntry}\pysiglinewithargsret{\strong{class }\code{Visualplotterwidget.}\bfcode{ParamEntry}}{\emph{parent=None}}{}
convenience container widget to hold parameters
\paragraph{Methods}

\end{fulllineitems}



\subsection{PopPlot}
\label{PopPlot::doc}\label{PopPlot:popplot}\index{PopPlot (class in Visualplotterwidget)}

\begin{fulllineitems}
\phantomsection\label{PopPlot:Visualplotterwidget.PopPlot}\pysiglinewithargsret{\strong{class }\code{Visualplotterwidget.}\bfcode{PopPlot}}{\emph{mod=None}, \emph{params=None}, \emph{do\_fit=False}, \emph{parent=None}}{}
popup class for plots both static and updating
\begin{quote}\begin{description}
\item[{Variables}] \leavevmode\begin{itemize}
\item {} 
\textbf{\texttt{ax}} -- matplotlib axis

\item {} 
\textbf{\texttt{figure}} -- matplotlib figure

\item {} 
\textbf{\texttt{canvas}} -- matplotlib canvas

\item {} 
\textbf{\texttt{toolbar}} -- matplotlib navigation toolbar

\end{itemize}

\item[{Parameters}] \leavevmode\begin{itemize}
\item {} 
\textbf{\texttt{mod}} (\emph{lmfit.Model}) -- lmfit Model object for fitting

\item {} 
\textbf{\texttt{do\_fit}} (\emph{Boolean}) -- Boolean if fitting should occur

\item {} 
\textbf{\texttt{params}} (\emph{lmfit.Parameters}) -- fit parameters

\end{itemize}

\end{description}\end{quote}
\paragraph{Methods}
\index{plot() (Visualplotterwidget.PopPlot method)}

\begin{fulllineitems}
\phantomsection\label{PopPlot:Visualplotterwidget.PopPlot.plot}\pysiglinewithargsret{\bfcode{plot}}{\emph{x}, \emph{y}, \emph{xl}, \emph{yl}, \emph{title}, \emph{std}}{}
plot the data with a new fit if do\_fit == True
\begin{quote}\begin{description}
\item[{Params x}] \leavevmode
x vector of points

\item[{Params y}] \leavevmode
y vector of points

\item[{Params xl}] \leavevmode
x label

\item[{Params yl}] \leavevmode
y label

\item[{Params title}] \leavevmode
title of plot

\item[{Params std}] \leavevmode
standard devation of points

\end{description}\end{quote}

\end{fulllineitems}

\index{update() (Visualplotterwidget.PopPlot method)}

\begin{fulllineitems}
\phantomsection\label{PopPlot:Visualplotterwidget.PopPlot.update}\pysiglinewithargsret{\bfcode{update}}{\emph{x}, \emph{y}, \emph{std=None}}{}
update the plots call the plot function
\begin{quote}\begin{description}
\item[{Params x}] \leavevmode
x vector of points

\item[{Params y}] \leavevmode
y vector of points

\item[{Params std}] \leavevmode
standard devation of points

\end{description}\end{quote}

\end{fulllineitems}

\index{update\_init() (Visualplotterwidget.PopPlot method)}

\begin{fulllineitems}
\phantomsection\label{PopPlot:Visualplotterwidget.PopPlot.update_init}\pysiglinewithargsret{\bfcode{update\_init}}{\emph{xl}, \emph{yl}, \emph{title}, \emph{ignore}, \emph{start}}{}
update the parameters to start
\begin{quote}\begin{description}
\item[{Parameters}] \leavevmode\begin{itemize}
\item {} 
\textbf{\texttt{title}} (\emph{string}) -- title of plot

\item {} 
\textbf{\texttt{xl}} (\emph{string}) -- x label

\item {} 
\textbf{\texttt{yl}} (\emph{string}) -- y label

\item {} 
\textbf{\texttt{start}} (\emph{int}) -- starting index

\end{itemize}

\end{description}\end{quote}

\end{fulllineitems}


\end{fulllineitems}



\subsection{VisualPlotter}
\label{VisualPlotter::doc}\label{VisualPlotter:visualplotter}\index{VisualPlotter (class in Visualplotterwidget)}

\begin{fulllineitems}
\phantomsection\label{VisualPlotter:Visualplotterwidget.VisualPlotter}\pysiglinewithargsret{\strong{class }\code{Visualplotterwidget.}\bfcode{VisualPlotter}}{\emph{parent=None}}{}
Class to choose plotting visually so it is easy.  Will also 
automatically update plots for every shot.  Can automatically fit on a 
single shot or updating shot basis.
\begin{quote}\begin{description}
\item[{Variables}] \leavevmode\begin{itemize}
\item {} 
{\hyperref[Options:Optionswidgets.Options.message]{\emph{\textbf{\texttt{message}}}}} -- pyqtSignal which can be transmitted to main message box

\item {} 
\textbf{\texttt{plots}} -- Dictionary to hold all the plots

\item {} 
\textbf{\texttt{data}} -- local copy of entire pandas dataframe

\item {} 
\textbf{\texttt{index}} -- index of shot

\item {} 
{\hyperref[MainWindow:SpinorMonitor.MainWindow.start]{\emph{\textbf{\texttt{start}}}}} -- start of plot region

\item {} 
{\hyperref[MainWindow:SpinorMonitor.MainWindow.end]{\emph{\textbf{\texttt{end}}}}} -- end of plot region

\item {} 
\textbf{\texttt{ignore\_list}} -- list of shots to ignore

\end{itemize}

\item[{Fit\_models}] \leavevmode
different models to fit too needs to be updated when models added

\end{description}\end{quote}
\paragraph{Methods}
\index{add\_fitting\_widgets() (Visualplotterwidget.VisualPlotter method)}

\begin{fulllineitems}
\phantomsection\label{VisualPlotter:Visualplotterwidget.VisualPlotter.add_fitting_widgets}\pysiglinewithargsret{\bfcode{add\_fitting\_widgets}}{}{}
function populates stacked box for each type of fit

\end{fulllineitems}

\index{avg\_data() (Visualplotterwidget.VisualPlotter method)}

\begin{fulllineitems}
\phantomsection\label{VisualPlotter:Visualplotterwidget.VisualPlotter.avg_data}\pysiglinewithargsret{\bfcode{avg\_data}}{}{}
average data and transform self.x\_data and self.y\_data
this is a really crappy algorithm, but it does the trick

\end{fulllineitems}

\index{do\_fit() (Visualplotterwidget.VisualPlotter method)}

\begin{fulllineitems}
\phantomsection\label{VisualPlotter:Visualplotterwidget.VisualPlotter.do_fit}\pysiglinewithargsret{\bfcode{do\_fit}}{}{}
do a fit

\end{fulllineitems}

\index{filter\_ignore() (Visualplotterwidget.VisualPlotter method)}

\begin{fulllineitems}
\phantomsection\label{VisualPlotter:Visualplotterwidget.VisualPlotter.filter_ignore}\pysiglinewithargsret{\bfcode{filter\_ignore}}{\emph{data}}{}
filter data, list of indices to remove
built list of indices not ignored

\end{fulllineitems}

\index{ignore\_update() (Visualplotterwidget.VisualPlotter method)}

\begin{fulllineitems}
\phantomsection\label{VisualPlotter:Visualplotterwidget.VisualPlotter.ignore_update}\pysiglinewithargsret{\bfcode{ignore\_update}}{}{}
update the ignore list parse out test

\end{fulllineitems}

\index{make\_title\_string() (Visualplotterwidget.VisualPlotter method)}

\begin{fulllineitems}
\phantomsection\label{VisualPlotter:Visualplotterwidget.VisualPlotter.make_title_string}\pysiglinewithargsret{\bfcode{make\_title\_string}}{}{}
make a title string

\end{fulllineitems}

\index{make\_updating\_title\_string() (Visualplotterwidget.VisualPlotter method)}

\begin{fulllineitems}
\phantomsection\label{VisualPlotter:Visualplotterwidget.VisualPlotter.make_updating_title_string}\pysiglinewithargsret{\bfcode{make\_updating\_title\_string}}{}{}
make a title string

\end{fulllineitems}

\index{message (Visualplotterwidget.VisualPlotter attribute)}

\begin{fulllineitems}
\phantomsection\label{VisualPlotter:Visualplotterwidget.VisualPlotter.message}\pysigline{\bfcode{message}}
\end{fulllineitems}

\index{plot\_clicked() (Visualplotterwidget.VisualPlotter method)}

\begin{fulllineitems}
\phantomsection\label{VisualPlotter:Visualplotterwidget.VisualPlotter.plot_clicked}\pysiglinewithargsret{\bfcode{plot\_clicked}}{}{}
function called when any plot option is called, sets the start and
end values

\end{fulllineitems}

\index{static\_plot() (Visualplotterwidget.VisualPlotter method)}

\begin{fulllineitems}
\phantomsection\label{VisualPlotter:Visualplotterwidget.VisualPlotter.static_plot}\pysiglinewithargsret{\bfcode{static\_plot}}{}{}
create new modal popup static plot

\end{fulllineitems}

\index{test\_fit() (Visualplotterwidget.VisualPlotter method)}

\begin{fulllineitems}
\phantomsection\label{VisualPlotter:Visualplotterwidget.VisualPlotter.test_fit}\pysiglinewithargsret{\bfcode{test\_fit}}{}{}
do a fit on the test plto

\end{fulllineitems}

\index{test\_plot() (Visualplotterwidget.VisualPlotter method)}

\begin{fulllineitems}
\phantomsection\label{VisualPlotter:Visualplotterwidget.VisualPlotter.test_plot}\pysiglinewithargsret{\bfcode{test\_plot}}{}{}
update the test plot

\end{fulllineitems}

\index{update\_plots() (Visualplotterwidget.VisualPlotter method)}

\begin{fulllineitems}
\phantomsection\label{VisualPlotter:Visualplotterwidget.VisualPlotter.update_plots}\pysiglinewithargsret{\bfcode{update\_plots}}{\emph{df}, \emph{index}}{}
Update the updating plots whose references are stored in self.plots
\begin{quote}\begin{description}
\item[{Params df}] \leavevmode
pandas dataframe holding data

\item[{Params index}] \leavevmode
index of shot

\end{description}\end{quote}

\end{fulllineitems}

\index{updating\_plot() (Visualplotterwidget.VisualPlotter method)}

\begin{fulllineitems}
\phantomsection\label{VisualPlotter:Visualplotterwidget.VisualPlotter.updating_plot}\pysiglinewithargsret{\bfcode{updating\_plot}}{}{}
create an updating plot and fill it with parameters 
gathered from current state of widgets

\end{fulllineitems}

\index{validate() (Visualplotterwidget.VisualPlotter method)}

\begin{fulllineitems}
\phantomsection\label{VisualPlotter:Visualplotterwidget.VisualPlotter.validate}\pysiglinewithargsret{\bfcode{validate}}{\emph{el}}{}
valid to make sure is a single integer or list comprehension
and turn list comprehensions into their equivalent
definee here since its an object method in QTGUI

\end{fulllineitems}

\index{var\_push() (Visualplotterwidget.VisualPlotter method)}

\begin{fulllineitems}
\phantomsection\label{VisualPlotter:Visualplotterwidget.VisualPlotter.var_push}\pysiglinewithargsret{\bfcode{var\_push}}{\emph{var\_list}}{}
add a list of variables to options

\end{fulllineitems}

\index{verbose\_avg() (Visualplotterwidget.VisualPlotter method)}

\begin{fulllineitems}
\phantomsection\label{VisualPlotter:Visualplotterwidget.VisualPlotter.verbose_avg}\pysiglinewithargsret{\bfcode{verbose\_avg}}{\emph{x}, \emph{y}}{}
average data and transform self.x\_data and self.y\_data
this is a really crappy algorithm, but it does the trick

\end{fulllineitems}


\end{fulllineitems}



\chapter{Indices and tables}
\label{index:indices-and-tables}\begin{itemize}
\item {} 
\DUspan{xref,std,std-ref}{genindex}

\item {} 
\DUspan{xref,std,std-ref}{modindex}

\item {} 
\DUspan{xref,std,std-ref}{search}

\end{itemize}



\renewcommand{\indexname}{Index}
\printindex
\end{document}
